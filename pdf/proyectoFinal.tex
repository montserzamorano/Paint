\input{preambuloSimple.tex}
\usepackage{url} %bibliografía
\usepackage{appendix}
\usepackage{sectsty}
\usepackage{listings}
\usepackage{xcolor}
\usepackage{colortbl}
\usepackage{breakurl}
%\usepackage{hyperref}
%\usepackage[breaklinks=true]{hyperref}
\usepackage[hidelinks]{hyperref} 
\usepackage{fancyvrb}
\usepackage{color}

\usepackage{multirow} % para las tablas


\usepackage{anysize}
\marginsize{3cm}{3cm}{2.5cm}{2.5cm} %izq, derecha, arriba
% Se limpia la cabecera y el pie de página para poder rehacerlos luego.

% Espacios en el documento:
\linespread{1}                        % Espacio entre líneas.
\setlength\parindent{0pt}               % Selecciona la indentación para cada inicio de párrafo.


\lstset{
        tabsize=2, % tab = 2 espacios
        backgroundcolor=\color[HTML]{F0F0F0}, % color de fondo
        captionpos=b, % posición de pie de código, b=debajo
        basicstyle=\ttfamily, % estilo de letra general
        columns=fixed, % columnas alineadas
        extendedchars=true, % ASCII extendido
        breaklines=true, % partir líneas
        %prebreak = \raisebox{0ex}[0ex][0ex]{\ensuremath{\hookleftarrow}}, % marcar final de línea con flecha
        showtabs=false, % no marcar tabulación
        showspaces=false, % no marcar espacios
        keywordstyle=\bfseries\color[HTML]{007020}, % estilo de palabras clave
        commentstyle=\itshape\color[HTML]{60A0B0}, % estilo de comentarios
        stringstyle=\color[HTML]{4070A0}, % estilo de strings
}

\renewcommand{\appendixname}{Anexos}
\renewcommand{\appendixtocname}{Anexos}
\renewcommand{\appendixpagename}{Anexos}

\title{	
\normalfont \normalsize 
\textsc{{\bf Aprendizaje Automático (2018-2019)} \\ Grado en Ingeniería Informática \\ Universidad de Granada} \\ [25pt] % Your university, school and/or dbrewepartment name(s)
\horrule{0.5pt} \\[0.4cm] % Thin top horizontal rule
\huge Evaluación Sistemas Multimedia \\ % The assignment title
\horrule{2pt} \\[0.5cm] % Thick bottom horizontal rule
}
\author{Montserrat Rodríguez Zamorano} % Nombre y apellidos
\date{\normalsize\today} % Incluye la fecha actual

\begin{document}
\pagestyle{empty}
\maketitle
\vskip1cm
\clearpage
\tableofcontents
\clearpage
%\vskip0.3cm
%\begin{figure}[H]
 % \centering
  %  \includegraphics[width=0.8\textwidth]{}
 % \caption{Diseño conceptual}
 % \label{diseño}
%\end{figure}
\setcounter{page}{1}
\pagestyle{plain}
\section{Descripción del sistema}
\section{Requisitos}
\subsection{Requisitos funcionales}
\subsubsection{Carácter general}
\begin{itemize}
\item{}
\end{itemize}
\subsubsection{Dibujo}
\begin{itemize}
\item{}
\end{itemize}
\subsubsection{Procesamiento de imágenes}
\begin{itemize}
\item{}
\end{itemize}
\subsubsection{Sonido}
\begin{itemize}
\item{}
\end{itemize}
\subsubsection{Vídeo}
\begin{itemize}
\item{}
\end{itemize}
\subsection{Requisitos no funcionales}
\section{Análisis}

\section{Diseño}

\section{Codificación}

%\section{Bonus}

%\begin{lstlisting}[language=python, caption=Código gradiente descendente, label=lst:graddesc]
%\end{lstlisting}
%\renewcommand{\labelenumi}{\alph{enumi})}

%%BIBLIOGRAFIA%%
%\clearpage
%\begin{thebibliography}{X}

%\end{thebibliography}
\end{document}