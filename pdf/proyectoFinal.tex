\input{preambuloSimple.tex}
\usepackage{url} %bibliografía
\usepackage{appendix}
\usepackage{sectsty}
\usepackage{listings}
\usepackage{xcolor}
\usepackage{colortbl}
\usepackage{breakurl}
%\usepackage{hyperref}
%\usepackage[breaklinks=true]{hyperref}
\usepackage[hidelinks]{hyperref} 
\usepackage{fancyvrb}
\usepackage{color}

\usepackage{multirow} % para las tablas


\usepackage{anysize}
\marginsize{3cm}{3cm}{2.5cm}{2.5cm} %izq, derecha, arriba
% Se limpia la cabecera y el pie de página para poder rehacerlos luego.

% Espacios en el documento:
\linespread{1}                        % Espacio entre líneas.
\setlength\parindent{0pt}               % Selecciona la indentación para cada inicio de párrafo.


\lstset{
        tabsize=2, % tab = 2 espacios
        backgroundcolor=\color[HTML]{F0F0F0}, % color de fondo
        captionpos=b, % posición de pie de código, b=debajo
        basicstyle=\ttfamily, % estilo de letra general
        columns=fixed, % columnas alineadas
        extendedchars=true, % ASCII extendido
        breaklines=true, % partir líneas
        %prebreak = \raisebox{0ex}[0ex][0ex]{\ensuremath{\hookleftarrow}}, % marcar final de línea con flecha
        showtabs=false, % no marcar tabulación
        showspaces=false, % no marcar espacios
        keywordstyle=\bfseries\color[HTML]{007020}, % estilo de palabras clave
        commentstyle=\itshape\color[HTML]{60A0B0}, % estilo de comentarios
        stringstyle=\color[HTML]{4070A0}, % estilo de strings
}

\renewcommand{\appendixname}{Anexos}
\renewcommand{\appendixtocname}{Anexos}
\renewcommand{\appendixpagename}{Anexos}

\title{	
\normalfont \normalsize 
\textsc{{\bf Aprendizaje Automático (2018-2019)} \\ Grado en Ingeniería Informática \\ Universidad de Granada} \\ [25pt] % Your university, school and/or dbrewepartment name(s)
\horrule{0.5pt} \\[0.4cm] % Thin top horizontal rule
\huge Evaluación Sistemas Multimedia \\ % The assignment title
\horrule{2pt} \\[0.5cm] % Thick bottom horizontal rule
}
\author{Montserrat Rodríguez Zamorano} % Nombre y apellidos
\date{\normalsize\today} % Incluye la fecha actual

\begin{document}
\pagestyle{empty}
\maketitle
\vskip1cm
\clearpage
\tableofcontents
\clearpage
%\vskip0.3cm
%\begin{figure}[H]
 % \centering
  %  \includegraphics[width=0.8\textwidth]{}
 % \caption{Diseño conceptual}
 % \label{diseño}
%\end{figure}
\setcounter{page}{1}
\pagestyle{plain}
\section{Descripción del sistema}
Para la evaluación de la asignatura de \textit{Sistemas multimedia} se quiere realizar una aplicación multimedia que permita gestionar gráficos, imágenes, sonido y vídeo. Se podrán crear, editar, procesar y visualizar contenido multimedia de distintos tipos.
\vskip0.3cm
Con este fin, se realiza una aplicación que se llamará \textit{Paint}.
\section{Requisitos}
\subsection{Requisitos funcionales}
\subsubsection{Carácter general}
\begin{itemize}
\item{\textbf{[RFCG-1]} Creación de una nueva imagen en una nueva ventana.}
\item{\textbf{[RFCG-2} Abrir un fichero de imagen.}
\item{\textbf{[RFCG-3]} Abrir un fichero de sonido.}
\item{\textbf{[RFCG-4]} Abrir un fichero de vídeo.}
\item{\textbf{[RFCG-5]} Guardar una imagen y sus figuras dibujadas.}
\item{\textbf{[RFCG-6]} Ocultar las barras de herramientas.}
\item{\textbf{[RFCG-7]} Visualizar las barras de herramientas.}
\item{\textbf{[RFCG-8]} Consultar el nombre del programa, versión y autor.}
\end{itemize}
\subsubsection{Dibujo}
\begin{itemize}
\item{\textbf{[RFD-1]} Dibujar las siguientes formas geométricas con sus propios atributos independientes.
\begin{itemize}
\item{Línea.}
\item{Rectángulo.}
\item{Elipse.}
\item{Rectángulo con esquinas redondeadas.}
\end{itemize}
}
\item{\textbf{[RFD-2]} Mantener todas las figuras que se vayan dibujando.}
\item{\textbf{[RFD-3]} Elegir el color de trazo de dibujo.
\begin{itemize}
\item{Rojo.}
\item{Azul.}
\item{Negro.}
\item{Blanco.}
\item{Verde.}
\end{itemize}
}
\item{\textbf{[RFD-4]} No rellenar la imagen.}
\item{\textbf{[RFD-5]} Rellenar con color el dibujo.
\begin{itemize}
\item{Rojo.}
\item{Azul.}
\item{Negro.}
\item{Blanco.}
\item{Verde.}
\end{itemize}
}
\item{\textbf{[RFD-6]} Seleccionar una figura dibujada.}
\item{\textbf{[RFD-7]} Editar una figura dibujada.}
\item{\textbf{[RFD-8]} Mover una figura dibujada.}
\item{\textbf{[RFD-9]} Consultar los atributos de una figura dibujada.}
\item{\textbf{[RFD-10]} Asociar un grado de transparencia a una figura.}
\item{\textbf{[RFD-11]} Activar alisado de bordes de una figura.}
\item{\textbf{[RFD-12]} Desactivar alisado de bordes de una figura.}
\end{itemize}
\subsubsection{Procesamiento de imágenes}
\begin{itemize}
\item{\textbf{[RFPI-1]} Duplicar una imagen.}
\item{\textbf{[RFPI-2]} Modificar el brillo de una imagen.}
\item{\textbf{[RFPI-3]} Aplicar filtro para emborronar una imagen.}
\item{\textbf{[RFPI-4]} Aplicar filtro para enfocar una imagen.}
\item{\textbf{[RFPI-5]} Aplicar filtro de relieve a una imagen.}
\item{\textbf{[RFPI-6]} Aplicar contraste a una imagen.}
\item{\textbf{[RFPI-7]} Iluminar una imagen.}
\item{\textbf{[RFPI-8]} Oscurecer una imagen.}
\item{\textbf{[RFPI-9]} Extraer las bandas de una imagen.}
\item{\textbf{[RFPI-10]} Invertir los colores de una imagen.}
\item{\textbf{[RFPI-11]} Convertir una imagen a los siguientes espacios:
\begin{itemize}
\item{RGB}
\item{YCC}
\item{GRAY}
\end{itemize}
}
\item{\textbf{[RFPI-12]} Girar una imagen a cualquier ángulo.}
\item{\textbf{[RFPI-13]} Tintar una imagen.}
\item{\textbf{[RFPI-14]} Escoger nivel de tintado de una imagen.}
\item{\textbf{[RFPI-15]} Ecualizar una imagen.}
\item{\textbf{[RFPI-16]} Umbralizar una imagen en niveles de gris.}
\item{\textbf{[RFPI-17]} Escoger nivel de umbralización.}
\item{\textbf{[RFPI-18]} Aplicar filtro sepia a una imagen.}
\item{\textbf{[RFPI-19]} Aplicar filtro violeta a una imagen.}
\item{\textbf{[RFPI-20]} Aplicar filtro de media entre las bandas de colores.}
\item{\textbf{[RFPI-21]} Aplicar filtro cosinosuidal.}
\end{itemize}
\subsubsection{Sonido}
\begin{itemize}
\item{\textbf{[RFS-1]} Reproducir audios.}
\item{\textbf{[RFS-2]} Grabar sonidos.}
\item{\textbf{[RFS-3]} Pausar la reproducción.}
\item{\textbf{[RFS-4]} Parar la grabación.}
\item{\textbf{[RFS-5]} Parar la reproducción.}
\end{itemize}
\subsubsection{Vídeo}
\begin{itemize}
\item{\textbf{[RFV-1]} Mostrar la secuencia que capte la Webcam.}
\item{\textbf{[RFV-2]} Capturar una imagen desde la Webcam.}
\end{itemize}
\subsection{Requisitos no funcionales}
\begin{itemize}
\item{\textbf{[RNF-1]} Se mostrará en la barra de estado el pixel en el que está situado el ratón.}
\item{\textbf{[RNF-2]} Se habilitarán en cada momento sólo los botones que pueden utilizarse. Por ejemplo, si se abre una ventana de vídeo se deshabilitarán aquellos correspondientes a las imágenes.}
\item{\textbf{[RNF-3]} Los botones tendrán asociados un \textit{ToolTipText} para facilitar su uso.}
\item{\textbf{[RNF-4]} Al seleccionar una figura se activarán sus propiedades en la barra de atributos.}
\item{\textbf{[RNF-5]} Si hay una figura seleccionada, al pulsar el ratón sobre en otro punto, deberá deseleccionarse la figura.}
\item{\textbf{[RNF-6]} El título de una nueva ventana abierta será el nombre del fichero si se trata de una imagen abierta o guardada.}
\item{\textbf{[RNF-7]} El título de una nueva ventana abierta será \textit{Nueva} si se trata de una imagen creada por el usuario.}
\item{\textbf{[RNF-8]} El título de una nueva ventana abierta será \textit{Captura} si se trata de una captura captada de un vídeo o de la WebCam.}
\item{\textbf{[RNF-9]} La \textit{BoundingBox} será un rectángulo de color azul, con línea discontinua.}
\item{\textbf{[RNF-10]} Cuando se cree una imagen se lanzará un diálogo que permita elegir las dimensiones de la imagen.}

\end{itemize}
\section{Análisis}

Se analizará qué requisitos son cubiertos por las bibliotecas disponibles y para cuáles necesitamos desarrollar soluciones alternativas.
\vskip0.3cm
En primer lugar, se desarrollará una clase \textit{Lienzo} que constituirá el área de dibujo y que almacenará las figuras dibujadas en una misma ventana. La clase \textit{Lienzo} será la encargada de pintar las figuras.
\vskip0.3cm
La solución planteada durante el desarrollo de las prácticas no es muy flexible: todas las figuras de un mismo lienzo tienen los mismos atributos. Para cumplir \textbf{[RFD-1]}-\textbf{[RFD-5]} y \textbf{[RFD-9]}-\textbf{[RFD-12]} se desarrollará una clase propia \textit{Figura}, que almacenará cada uno de los atributos de la figura y permitirá que cada figura tenga los suyos propios. El uso de esta clase permitirá además modificar estos atributos \textbf{[RFD-7]}. Sin embargo, la línea, por ejemplo, no puede tener relleno, por lo que no tiene sentido guardar este atributo.
\vskip0.3cm
Para solucionarlo, se desarrollará una clase propia para cada una de las figuras que pueden dibujarse: Línea, rectángulo, elipse, rectángulo redondeado. El uso de estas clases permitirá mover las figuras dibujadas \textbf{[RFD-8]},implementando en cada caso la edición de la posición.
\vskip0.3cm
Esta solución completa los requisitos del módulo de figuras. Para el caso del procesamiento de imágenes, se tendrán que desarrollar clases propias para cumplir con \textbf{[RFPI-16]}-\textbf{[RFPI-20]}. El resto de los requisitos funcionales están cubiertos por las funciones ofrecidas por Java y las bibliotecas proporcionadas por el profesor. Para el caso de cosinosuidal \textbf{[RFPI-21]} se hará uso de la clase \textit{LookUpTable} con una operación propia (coseno).
\vskip0.3cm
En cuanto al manejo de eventos, se usará en muchas ocasiones las funciones que ofrece Java para la gestión de eventos (por ejemplo, para el uso de botones). Sin embargo, para la comunicación entre el lienzo y la ventana principal será necesario crear una clase manejadora para la gestión de eventos relacionados con el lienzo. La existencia de esta clase permitirá también informar de la falta de atributos necesarios para el dibujo de las figuras y poder lanzar mensajes de error desde la ventana principal (por ejemplo: no se ha seleccionado una forma de dibujo). 
\section{Diseño}
Siguiendo la propuesta que se ha planteado en el análisis, se plantea la siguiente jerarquía de clases.
\section{Implementación}
\section{Validación software}
\subsection{Gráficos}
\subsection{Procesamiento de imágenes}
\subsubsection{Duplicar}
Se comprobará si la duplicación de la imagen se ha realizado correctamente, de forma que si se modifica una, los cambios no se aplican en la otra.
\subsubsection{Modificar brillo}
\vskip0.3cm
\begin{figure}[H]
 \centering
  \includegraphics[width=0.3\textwidth]{imagenes/fryBrilo2.jpg}
  \includegraphics[width=0.3\textwidth]{imagenes/Fry.jpg}
  \includegraphics[width=0.3\textwidth]{imagenes/fryBrillo1.jpg}
 \caption{Modificación del brillo}
 \label{diseño}
\end{figure}
\subsubsection{Filtro de emborronamiento}
Se ha aplicado varias veces el filtro para que sea visible.
\vskip0.3cm
\begin{figure}[H]
 \centering
  \includegraphics[width=0.3\textwidth]{imagenes/Fry.jpg}
  \includegraphics[width=0.3\textwidth]{imagenes/fryEmborronamiento.jpg}
 \caption{Aplicación filtro emborronamiento}
 \label{diseño}
\end{figure}
\subsubsection{Filtro de enfoque}
\vskip0.3cm
\begin{figure}[H]
 \centering
  \includegraphics[width=0.3\textwidth]{imagenes/Fry.jpg}
  \includegraphics[width=0.3\textwidth]{imagenes/fryEnfoque.jpg}
 \caption{Aplicación filtro enfoque}
 \label{diseño}
\end{figure}
\subsubsection{Filtro de relieve}
\vskip0.3cm
\begin{figure}[H]
 \centering
  \includegraphics[width=0.3\textwidth]{imagenes/Fry.jpg}
  \includegraphics[width=0.3\textwidth]{imagenes/fryRelieve.jpg}
 \caption{Aplicación filtro relieve}
 \label{diseño}
\end{figure}
\subsubsection{Contraste normal}
\vskip0.3cm
\begin{figure}[H]
 \centering
  \includegraphics[width=0.3\textwidth]{imagenes/Fry.jpg}
  \includegraphics[width=0.3\textwidth]{imagenes/fryContraste.jpg}
 \caption{Aplicación contraste}
 \label{diseño}
\end{figure}
\subsubsection{Iluminado}
\vskip0.3cm
\begin{figure}[H]
 \centering
  \includegraphics[width=0.3\textwidth]{imagenes/Fry.jpg}
  \includegraphics[width=0.3\textwidth]{imagenes/fryIluminado.jpg}
 \caption{Aplicación iluminación}
 \label{diseño}
\end{figure}
\subsubsection{Oscurecido}
\vskip0.3cm
\begin{figure}[H]
 \centering
  \includegraphics[width=0.3\textwidth]{imagenes/Fry.jpg}
  \includegraphics[width=0.3\textwidth]{imagenes/fryOscurecido.jpg}
 \caption{Aplicación oscurecido}
 \label{diseño}
\end{figure}
\subsubsection{Invertir colores}
\vskip0.3cm
\begin{figure}[H]
 \centering
  \includegraphics[width=0.3\textwidth]{imagenes/Fry.jpg}
  \includegraphics[width=0.3\textwidth]{imagenes/fryNegativo.jpg}
 \caption{Aplicación filtro negativo}
 \label{diseño}
\end{figure}
\subsubsection{Conversión a espacios RGB, YCC, GRAY}
\vskip0.3cm
\begin{figure}[H]
 \centering
  \includegraphics[width=0.3\textwidth]{imagenes/banda3RGB.jpg}
  \includegraphics[width=0.3\textwidth]{imagenes/banda3YCC.jpg}
 \caption{}
 \label{diseño}
\end{figure}
\subsubsection{Giro libre}
\subsubsection{Escalado}
\subsubsection{Tintado}
\vskip0.3cm
\begin{figure}[H]
 \centering
  \includegraphics[width=0.3\textwidth]{imagenes/fryTinteRojo.jpg}
  \includegraphics[width=0.3\textwidth]{imagenes/Fry.jpg}
  \includegraphics[width=0.3\textwidth]{imagenes/fryTinteAmarillo.jpg}
 \caption{Tintado de la imagen}
 \label{diseño}
\end{figure}
\subsubsection{Ecualización}
\subsubsection{Filtro sepia}
\vskip0.3cm
\begin{figure}[H]
 \centering
  \includegraphics[width=0.3\textwidth]{imagenes/Fry.jpg}
  \includegraphics[width=0.3\textwidth]{imagenes/frySepia.jpg}
 \caption{Aplicación filtro sepia}
 \label{diseño}
\end{figure}
\subsubsection{Umbralización}
\vskip0.3cm
\begin{figure}[H]
 \centering
  \includegraphics[width=0.3\textwidth]{imagenes/Fry.jpg}
  \includegraphics[width=0.3\textwidth]{imagenes/fryUmbralizado.jpg}
 \caption{Umbralizar imagen}
 \label{diseño}
\end{figure}
\subsubsection{Operador \textit{LookupOp} basado en una función propia}
\subsubsection{Operación de diseño propio: filtro violeta}
\subsubsection{Operación de diseño propio: media de las bandas}
\subsection{Sonido}
\subsection{Vídeo}
%\section{Bonus}

%\begin{lstlisting}[language=python, caption=Código gradiente descendente, label=lst:graddesc]
%\end{lstlisting}
%\renewcommand{\labelenumi}{\alph{enumi})}

%%BIBLIOGRAFIA%%
%\clearpage
%\begin{thebibliography}{X}

%\end{thebibliography}
\end{document}