\input{preambuloSimple.tex}
\usepackage{url} %bibliografía
\usepackage{appendix}
\usepackage{sectsty}
\usepackage{listings}
\usepackage{xcolor}
\usepackage{colortbl}
\usepackage{breakurl}
%\usepackage{hyperref}
%\usepackage[breaklinks=true]{hyperref}
\usepackage[hidelinks]{hyperref} 
\usepackage{fancyvrb}
\usepackage{color}

\usepackage{multirow} % para las tablas


\usepackage{anysize}
\marginsize{3cm}{3cm}{2.5cm}{2.5cm} %izq, derecha, arriba
% Se limpia la cabecera y el pie de página para poder rehacerlos luego.

% Espacios en el documento:
\linespread{1}                        % Espacio entre líneas.
\setlength\parindent{0pt}               % Selecciona la indentación para cada inicio de párrafo.


\lstset{
        tabsize=2, % tab = 2 espacios
        backgroundcolor=\color[HTML]{F0F0F0}, % color de fondo
        captionpos=b, % posición de pie de código, b=debajo
        basicstyle=\ttfamily, % estilo de letra general
        columns=fixed, % columnas alineadas
        extendedchars=true, % ASCII extendido
        breaklines=true, % partir líneas
        %prebreak = \raisebox{0ex}[0ex][0ex]{\ensuremath{\hookleftarrow}}, % marcar final de línea con flecha
        showtabs=false, % no marcar tabulación
        showspaces=false, % no marcar espacios
        keywordstyle=\bfseries\color[HTML]{007020}, % estilo de palabras clave
        commentstyle=\itshape\color[HTML]{60A0B0}, % estilo de comentarios
        stringstyle=\color[HTML]{4070A0}, % estilo de strings
}

\renewcommand{\appendixname}{Anexos}
\renewcommand{\appendixtocname}{Anexos}
\renewcommand{\appendixpagename}{Anexos}

\title{	
\normalfont \normalsize 
\textsc{{\bf Aprendizaje Automático (2018-2019)} \\ Grado en Ingeniería Informática \\ Universidad de Granada} \\ [25pt] % Your university, school and/or dbrewepartment name(s)
\horrule{0.5pt} \\[0.4cm] % Thin top horizontal rule
\huge Evaluación Sistemas Multimedia \\ % The assignment title
\horrule{2pt} \\[0.5cm] % Thick bottom horizontal rule
}
\author{Montserrat Rodríguez Zamorano} % Nombre y apellidos
\date{\normalsize\today} % Incluye la fecha actual

\begin{document}
\pagestyle{empty}
\maketitle
\vskip1cm
\clearpage
\tableofcontents
\clearpage
%\vskip0.3cm
%\begin{figure}[H]
 % \centering
  %  \includegraphics[width=0.8\textwidth]{}
 % \caption{Diseño conceptual}
 % \label{diseño}
%\end{figure}
\setcounter{page}{1}
\pagestyle{plain}
\section{Descripción del sistema}
Para la evaluación de la asignatura de \textit{Sistemas multimedia} se quiere realizar una aplicación multimedia que permita gestionar gráficos, imágenes, sonido y vídeo. Se podrán crear, editar, procesar y visualizar contenido multimedia de distintos tipos.
\vskip0.3cm
Con este fin, se realiza una aplicación que se llamará \textit{Paint}.
\section{Requisitos}
\subsection{Requisitos funcionales}
\subsubsection{Carácter general}
\begin{itemize}
\item{\textbf{[RFCG-1]} Creación de una nueva imagen en una nueva ventana en un tamaño que elija el usuario.}
\item{\textbf{[RFCG-2} Abrir un fichero de imagen.}
\item{\textbf{[RFCG-3]} Abrir un fichero de sonido.}
\item{\textbf{[RFCG-4]} Abrir un fichero de vídeo.}
\item{\textbf{[RFCG-5]} Guardar una imagen y sus figuras dibujadas.}
\item{\textbf{[RFCG-6]} Ocultar las barras de herramientas.}
\item{\textbf{[RFCG-7]} Visualizar las barras de herramientas.}
\item{\textbf{[RFCG-8]} Consultar el nombre del programa, versión y autor.}
\end{itemize}
\subsubsection{Dibujo}
\begin{itemize}
\item{\textbf{[RFD-1]} Dibujar las siguientes formas geométricas con sus propios atributos independientes.
\begin{itemize}
\item{Línea.}
\item{Rectángulo.}
\item{Elipse.}
\end{itemize}
}
\item{\textbf{[RFD-2]} Mantener todas las figuras que se vayan dibujando.}
\item{\textbf{[RFD-3]} Elegir el color de trazo de dibujo.
\begin{itemize}
\item{Rojo.}
\item{Azul.}
\item{Negro.}
\item{Blanco.}
\item{Verde.}
\end{itemize}
}
\item{\textbf{[RFD-4]} No rellenar la imagen.}
\item{\textbf{[RFD-5]} Rellenar con color el dibujo.
\begin{itemize}
\item{Rojo.}
\item{Azul.}
\item{Negro.}
\item{Blanco.}
\item{Verde.}
\end{itemize}
}
\item{\textbf{[RFD-6]} Seleccionar una figura dibujada.}
\item{\textbf{[RFD-7]} Editar una figura dibujada.}
\item{\textbf{[RFD-8]} Mover una figura dibujada.}
\item{\textbf{[RFD-9]} Consultar los atributos de una figura dibujada.}
\item{\textbf{[RFD-10]} Asociar un grado de transparencia a una figura.}
\item{\textbf{[RFD-11]} Activar alisado de bordes de una figura.}
\item{\textbf{[RFD-12]} Desactivar alisado de bordes de una figura.}
\end{itemize}
\subsubsection{Procesamiento de imágenes}
\begin{itemize}
\item{\textbf{[RFPI-1]} Duplicar una imagen.}
\item{\textbf{[RFPI-2]} Modificar el brillo de una imagen.}
\item{\textbf{[RFPI-3]} Aplicar filtro para emborronar una imagen.}
\item{\textbf{[RFPI-4]} Aplicar filtro para enfocar una imagen.}
\item{\textbf{[RFPI-5]} Aplicar filtro de relieve a una imagen.}
\item{\textbf{[RFPI-6]} Aplicar contraste a una imagen.}
\item{\textbf{[RFPI-7]} Iluminar una imagen.}
\item{\textbf{[RFPI-8]} Oscurecer una imagen.}
\item{\textbf{[RFPI-9]} Extraer las bandas de una imagen.}
\item{\textbf{[RFPI-10]} Invertir los colores de una imagen.}
\item{\textbf{[RFPI-11]} Convertir una imagen a los siguientes espacios:
\begin{itemize}
\item{RGB}
\item{YCC}
\item{GRAY}
\end{itemize}
}
\item{\textbf{[RFPI-12]} Girar una imagen a cualquier ángulo.}
\item{\textbf{[RFPI-13]} Tintar una imagen.}
\item{\textbf{[RFPI-14]} Escoger nivel de tintado de una imagen.}
\item{\textbf{[RFPI-15]} Ecualizar una imagen.}
\item{\textbf{[RFPI-16]} Umbralizar una imagen en niveles de gris.}
\item{\textbf{[RFPI-17]} Escoger nivel de umbralización.}
\item{\textbf{[RFPI-18]} Aplicar filtro sepia a una imagen.}
\item{\textbf{[RFPI-19]} Aplicar filtro violeta a una imagen.}
\item{\textbf{[RFPI-20]} Aplicar filtro blanco y negro.}

\end{itemize}
\subsubsection{Sonido}
\begin{itemize}
\item{\textbf{[RFS-1]} Reproducir audios.}
\item{\textbf{[RFS-2]} Grabar sonidos.}
\item{\textbf{[RFS-3]} Pausar la reproducción.}
\item{\textbf{[RFS-4]} Parar la grabación.}
\item{\textbf{[RFS-5]} Parar la reproducción.}
\end{itemize}
\subsubsection{Vídeo}
\begin{itemize}
\item{\textbf{[RFV-1]} Mostrar la secuencia que capte la Webcam.}
\item{\textbf{[RFV-2]} Capturar una imagen desde la Webcam.}
\end{itemize}
\subsection{Requisitos no funcionales}
\begin{itemize}
\item{\textbf{[RNF-1]} Se mostrará en la barra de estado el pixel en el que está situado el ratón.}
\item{\textbf{[RNF-2]} Se habilitarán en cada momento sólo los botones que pueden utilizarse. Por ejemplo, si se abre una ventana de vídeo se deshabilitarán aquellos correspondientes a las imágenes.}
\item{\textbf{[RNF-3]} Los botones tendrán asociados un \textit{ToolTipText} para facilitar su uso.}
\item{\textbf{[RNF-4]} Al seleccionar una figura se activarán sus propiedades en la barra de atributos.}
\item{\textbf{[RNF-5]} Si hay una figura seleccionada, al pulsar el ratón sobre en otro punto, deberá deseleccionarse la figura.}
\item{\textbf{[RNF-6]} El título de una nueva ventana abierta será el nombre del fichero si se trata de una imagen abierta o guardada.}
\item{\textbf{[RNF-7]} El título de una nueva ventana abierta será \textit{Nueva} si se trata de una imagen creada por el usuario.}
\item{\textbf{[RNF-8]} El título de una nueva ventana abierta será \textit{Captura} si se trata de una captura captada de un vídeo o de la WebCam.}
\item{\textbf{[RNF-9]} La \textit{BoundingBox} será un rectángulo de color azul, con línea discontinua.}
\item{\textbf{[RNF-10]} Indicar el espacio de color en el que se encuentra la imagen.}

\end{itemize}
\section{Análisis}
\subsection{Eventos}
\section{Diseño}
Siguiendo la propuesta que se ha planteado en el análisis, se plantea la siguiente jerarquía de clases.
\section{Implementación}
\section{Validación software}
%\section{Bonus}

%\begin{lstlisting}[language=python, caption=Código gradiente descendente, label=lst:graddesc]
%\end{lstlisting}
%\renewcommand{\labelenumi}{\alph{enumi})}

%%BIBLIOGRAFIA%%
%\clearpage
%\begin{thebibliography}{X}

%\end{thebibliography}
\end{document}